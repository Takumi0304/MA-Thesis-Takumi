% 段落インデントと段落間の空白をAPA仕様に設定
\usepackage{indentfirst} 
\setlength{\parindent}{1.27cm} 
\setlength{\parskip}{0pt}      

% パッケージ
\usepackage{titlesec}
\usepackage{fancyhdr}
\usepackage{linguex}
\PassOptionsToPackage{table}{xcolor}
\usepackage{xcolor}
\usepackage{tabularx}


% セクションタイトルのフォーマット(APA第7版に準拠)

% Level 1: Section (Flush left, Bold, Large)
\titleformat{\section}[block]
  {\normalfont\bfseries\fontsize{26}{28}\selectfont}
  {\thesection}{1em}{}

\titlespacing*{\section}{0pt}{3.5ex plus 1ex minus .2ex}{2.3ex plus .2ex}

% Level 2: Subsection (Flush left, Bold)
\titleformat{\subsection}[block]
  {\normalfont\bfseries\fontsize{18}{22}\selectfont}
  {\thesubsection}{1em}{}

\titlespacing*{\subsection}{0pt}{3.0ex plus 1ex minus .2ex}{2.0ex plus .2ex}

% Level 3: Subsubsection (Flush left, Bold Italic)
\titleformat{\subsubsection}[block]
  {\normalfont\bfseries\itshape\fontsize{14}{16}\selectfont}
  {\thesubsubsection}{1em}{}

\titlespacing*{\subsubsection}{0pt}{2.5ex plus 1ex minus .2ex}{2.0ex plus .2ex}

% Level 4: Paragraph (Indented, Bold, run-in heading, ends with period)
\titleformat{\paragraph}[runin]
  {\normalfont\bfseries\normalsize}
  {\theparagraph}{1em}{}
  
\titlespacing*{\paragraph}{1.27cm}{3.0ex plus 1ex minus .2ex}{1em}

% Level 5: Subparagraph (Indented, Bold Italic, run-in heading, ends with period)
\titleformat{\subparagraph}[runin]
  {\normalfont\bfseries\itshape\normalsize}
  {\thesubparagraph}{1em}{}

\titlespacing*{\subparagraph}{1.27cm}{3.0ex plus 1ex minus .2ex}{1em}

% 標準文字サイズ設定(必要なら)
\renewcommand{\normalsize}{\fontsize{12}{16.5}\selectfont}

% fancyhdr設定
\usepackage{fancyhdr}
\pagestyle{fancy}
\fancyhf{}  
\fancyhead[L]{\nouppercase{\leftmark}}  
\fancyfoot[C]{\thepage} 

% セクションタイトルを leftmark に反映させる設定
\makeatletter
\renewcommand{\sectionmark}[1]{\markboth{\thesection\ #1}{}}
\makeatother
